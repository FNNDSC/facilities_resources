%% Preamble %%
%% A minimal LaTeX preamble
%% Some packates are needed to implement
%% Asciidoc features

\documentclass[11pt]{amsart}
\usepackage{geometry}                % See geometry.pdf to learn the layout options. There are lots.
\geometry{letterpaper}               % ... or a4paper or a5paper or ...
%\geometry{landscape}                % Activate for for rotated page geometry
%\usepackage[parfill]{parskip}       % Activate to begin paragraphs with an empty line rather than an indent

\usepackage{tcolorbox}
\usepackage{lipsum}

\usepackage{epstopdf}
\usepackage{color}
% \usepackage[usenames, dvipsnames]{color}
% \usepackage{alltt}


\usepackage{amssymb}
% \usepackage{amsmath}
\usepackage{amsthm}
\usepackage[version=3]{mhchem}


% Needed to properly typeset
% standard unicode characters:
%
\RequirePackage{fix-cm}
\usepackage{fontspec}
\usepackage[Latin,Greek]{ucharclasses}
%
% NOTE: you must also use xelatex
% as the typesetting engine


% \usepackage{fontspec}
% \usepackage{polyglossia}
% \setmainlanguage{en}

\usepackage{hyperref}
\hypersetup{
    colorlinks=true,
    linkcolor=blue,
    filecolor=magenta,
    urlcolor=cyan,
}

\usepackage{graphicx}
\usepackage{wrapfig}
\graphicspath{ {images/} }
\DeclareGraphicsExtensions{.png, .jpg, jpeg, .pdf}

%% \DeclareGraphicsRule{.tif}{png}{.png}{`convert #1 `dirname #1`/`basename #1 .tif`.png}
%% Asciidoc TeX Macros %%


% \pagecolor{black}
%%%%%%%%%%%%


% Needed for Asciidoc

\newcommand{\admonition}[2]{\textbf{#1}: {#2}}
\newcommand{\rolered}[1]{ \textcolor{red}{#1} }
\newcommand{\roleblue}[1]{ \textcolor{blue}{#1} }

\newtheorem{theorem}{Theorem}
\newtheorem{proposition}{Proposition}
\newtheorem{corollary}{Corollary}
\newtheorem{lemma}{Lemma}
\newtheorem{definition}{Definition}
\newtheorem{conjecture}{Conjecture}
\newtheorem{problem}{Problem}
\newtheorem{exercise}{Exercise}
\newtheorem{example}{Example}
\newtheorem{note}{Note}
\newtheorem{joke}{Joke}
\newtheorem{objection}{Objection}





%%%%%%%%%%%%%%%%%%%%%%%%%%%%%%%%%%%%%%%%%%%%%%%%%%%%%%%

%  Extended quote environment with author

\renewenvironment{quotation}
{   \leftskip 4em \begin{em} }
{\end{em}\par }

\def\signed#1{{\leavevmode\unskip\nobreak\hfil\penalty50\hskip2em
  \hbox{}\nobreak\hfil\raise-3pt\hbox{(#1)}%
  \parfillskip=0pt \finalhyphendemerits=0 \endgraf}}


\newsavebox\mybox

\newenvironment{aquote}[1]
  {\savebox\mybox{#1}\begin{quotation}}
  {\signed{\usebox\mybox}\end{quotation}}

\newenvironment{tquote}[1]
  {  {\bf #1} \begin{quotation} \\ }
  { \end{quotation} }

%% BOXES: http://tex.stackexchange.com/questions/83930/what-are-the-different-kinds-of-boxes-in-latex
%% ENVIRONMENTS: https://www.sharelatex.com/learn/Environments

\newenvironment{asciidocbox}
  {\leftskip6em\rightskip6em\par}
  {\par}

\newenvironment{titledasciidocbox}[1]
  {\leftskip6em\rightskip6em\par{\bf #1}\vskip-0.6em\par}
  {\par}



%%%%%%%%%%%%%%%%%%%%%%%%%%%%%%%%%%%%%%%%%%%%%%%%%%%%%%%%

%% http://texblog.org/tag/rightskip/


\newenvironment{preamble}
  {}
  {}

%% http://tex.stackexchange.com/questions/99809/box-or-sidebar-for-additional-text
%%
\newenvironment{sidebar}[1][r]
  {\wrapfigure{#1}{0.5\textwidth}\tcolorbox}
  {\endtcolorbox\endwrapfigure}


%%%%%%%%%%

\newenvironment{comment*}
  {\leftskip6em\rightskip6em\par}
  {\par}

  \newenvironment{remark*}
  {\leftskip6em\rightskip6em\par}
  {\par}


%% Dummy environment for testing:

\newenvironment{foo}
  {\bf Foo.\ }
  {}


\newenvironment{foo*}
  {\bf Foo.\ }
  {}


\newenvironment{click}
  {\bf Click.\ }
  {}

\newenvironment{click*}
  {\bf Click.\ }
  {}


\newenvironment{remark}
  {\bf Remark.\ }
  {}

\newenvironment{capsule}
  {\leftskip10em\par}
  {\par}

%%%%%%%%%%%%%%%%%%%%%%%%%%%%%%%%%%%%%%%%%%%%%%%%%%%%%

%% Style

\parindent0pt
\parskip8pt
%% User Macros %%
%% Front Matter %%

\title{Facilities and Other Resources}
\author{}
\date{}


%% Begin Document %%

\begin{document}
\maketitle
\hypertarget{x-harvard-medical-school}{\section*{Harvard Medical School}}
\hypertarget{x-overview}{\subsection*{Overview}}
O${}_{2}$ is a platform for Linux-based high-performance computing at Harvard Medical School that is readily available to BCH employees. The name is derived from being the next generation of the HMS "Orchestra" cluster, hence "O"${}_{2}$. O${}_{2}$ is managed by the Research Computing Group, part of HMS IT. O${}_{2}$ is an HPC cluster built on Linux and the \texttt{SLURM} open source job scheduler.


\hypertarget{x-detail}{\subsection*{Detail}}
O${}_{2}$ currently includes 371 computing nodes for a total of 11356 cores and ~98TB of memory:


\begin{itemize}

\item 232 nodes, each node hostname is composed by the prefix compute-a-16- or compute-a-17- and the node number, for example compute-a-16-28, compute-a-16-29, …​, compute-a-16-171. Each node has 32 physical compute cores, 256GB of memory and is connected to the network with a 10Gb ethernet card and in addition with a 40Gb Infiniband card.

\item 69 nodes, each node hostname is composed by the prefix compute-e-16- and the node number. Each node has 28 physical compute cores, 256GB of memory and is connected to the network with a 10Gb ethernet card.

\item 17 nodes, each node hostname is composed by the prefix compute-f-16- and the node number. Each node has 20 physical compute cores, 188GB of memory and is connected to the network with a 10Gb ethernet card.

\item 11 heterogenous high memory nodes, each node hostname is composed by the prefix compute-h-16- and the node number; 7 nodes have 750GB of memory, 1 node 300GB and the other node 1TB.

\item 6 GPU compute nodes, each node hostname is composed by the prefix compute-g-16- and the node number. Two nodes have 4 Tesla V100 each, two nodes have 8 Tesla K-80 each and two nodes have 4 Tesla M-40 for a total of 32 GPU units.

\item 3 transfer nodes, each node hostname is composed by the prefix compute-t-16- and the node number. Each node is a VM with 4 cores and 6GB of memory, those nodes are intended for data transfer to/from the /files filesystem.

\item 31 contributed nodes, for a total of 1160 cores and ~10TB of memory.

\end{itemize}


\end{document}

